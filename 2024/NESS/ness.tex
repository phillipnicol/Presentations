\documentclass[aspectratio=43]{beamer}
\usepackage{amsmath, amsthm, amssymb}
\usepackage{xcolor}

\newcommand{\R}{\mathbb{R}}


\title{Dimension reduction for single-cell and spatial RNA-seq using generalized bilinear models}
\author{\texorpdfstring{Phillip Nicol \\ Department of Biostatistics, Harvard University}{}}
\institute{NESS 2024 (Storrs, CT)}
\date{May 24, 2024}

\begin{document}

\titlepage

\begin{frame}{Introduction}

\end{frame}

\begin{frame}{PCA cannot be directly applied due to heterogeneous variances}

\end{frame}

\begin{frame}{Standard approach is to transform the counts prior to PCA}

The counts are commonly pre-processed by computing the \textit{Pearson residual}:

\begin{equation}
Z_{ij} := \frac{Y_{ij} - \hat{\mu}_{i}}{\sqrt{ \hat{\mu}_{i} - \hat{\mu}_{i}^2 / \hat{\alpha}_i}}
\end{equation}

\begin{itemize}
\item scTransform (SCT) (REF): Estimate $\hat{\mu}$ and $\hat{\alpha}$ with a NB GLM.
\item APR (REF): Fix $\hat{\alpha} = 100$ (XX) and use a closed-form approximation to $\hat{\mu}$. 
\end{itemize}
\end{frame}

\begin{frame}{PCA on Pearson residuals can fail to capture rare cell types}
If baseline mean is large then $Z_{ij} \approx Z_{ij'}$ even for very different counts.

FIG XXX 
\end{frame}

\begin{frame}{ERCC scaling}

\end{frame}

\begin{frame}{scGBM simultaneously models the counts and reduces dimensionality}

\begin{equation}
\begin{split}
Y_{ij} &\sim \text{Pois}(\mu_{ij}) \\
\log(\mu_{ij}) &= \alpha_i + \beta_j + \sum_{m=1}^M \sigma_m u_{im} v_{jm} \\
\sigma_m &\sim \text{Expo}(a)
\end{split}
\end{equation}

$V := [v_{im}] \in \R^{J \times M}$ is the (low-dimensional) cell embeddings.


FIG XX 
\end{frame}

\begin{frame}{Estimation with iteratively reweighted singular value decomposition}
Define $\hat{X}^{(t)}$ to be the current estimate of $\hat{U} \hat{\Sigma} \hat{V}^{\top}$. The following update is used for the latent factors:

\begin{equation}
\hat{X}^{(t+1)} = \text{SVD}_{M, 1/a} \left( \hat{X}^{(t)} + \gamma( Y - \hat{\mu})     \right) 
\end{equation}
$\text{SVD}_{M, 1/a}(\cdot)$ computes the rank $M$ truncated SVD and then soft-thresholds the remaining singular values by $1/a$. 

\end{frame}

\begin{frame}{Faster estimation using scGBM-proj}
When $J$ is very large, first estimate $\hat{\alpha}, \hat{U}$ using a smaller subset of cells.

\vspace{3em}

Then holding $\hat{\alpha}$ and $\hat{U}$ fixed, the parameters $\beta$ and $V \Sigma$ can be estimated by fitting $J$ GLMs in parallel. 

\vspace{3em}

By analogy to PCA, we call this the \textit{projection method} (scGBM-proj)

\end{frame}

\begin{frame}{scGBM is faster and more accurate than GLM-PCA}

\end{frame}

\begin{frame}{Single marker genes}

\end{frame}

\begin{frame}{ERCC Scaling}

\end{frame}

\begin{frame}{scGBM quantifies uncertainty in the low-dimensional embedding of cells}

\end{frame}

\begin{frame}{Cluster confidence index}

\end{frame}

\begin{frame}{Extending to spatial transcriptomics}

\end{frame}

\begin{frame}{Edge-aware spatial smoothing}

\end{frame}

\begin{frame}{Acknowledgements}
\begin{itemize}
\item Jeff Miller (Harvard Biostatistics)
\item Rafa Irizarry (DFCI Data Science)
\item NIH Cancer Training Grant (REF) 
\end{itemize}
\end{frame}


\end{document}